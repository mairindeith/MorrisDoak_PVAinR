\documentclass[]{article}
\usepackage{lmodern}
\usepackage{amssymb,amsmath}
\usepackage{ifxetex,ifluatex}
\usepackage{fixltx2e} % provides \textsubscript
\ifnum 0\ifxetex 1\fi\ifluatex 1\fi=0 % if pdftex
  \usepackage[T1]{fontenc}
  \usepackage[utf8]{inputenc}
\else % if luatex or xelatex
  \ifxetex
    \usepackage{mathspec}
  \else
    \usepackage{fontspec}
  \fi
  \defaultfontfeatures{Ligatures=TeX,Scale=MatchLowercase}
\fi
% use upquote if available, for straight quotes in verbatim environments
\IfFileExists{upquote.sty}{\usepackage{upquote}}{}
% use microtype if available
\IfFileExists{microtype.sty}{%
\usepackage{microtype}
\UseMicrotypeSet[protrusion]{basicmath} % disable protrusion for tt fonts
}{}
\usepackage[margin=1in]{geometry}
\usepackage{hyperref}
\hypersetup{unicode=true,
            pdftitle={Chapter 2 - The Causes and Quantification of Population Vulnerability},
            pdfauthor={Mairin Deith},
            pdfborder={0 0 0},
            breaklinks=true}
\urlstyle{same}  % don't use monospace font for urls
\usepackage{color}
\usepackage{fancyvrb}
\newcommand{\VerbBar}{|}
\newcommand{\VERB}{\Verb[commandchars=\\\{\}]}
\DefineVerbatimEnvironment{Highlighting}{Verbatim}{commandchars=\\\{\}}
% Add ',fontsize=\small' for more characters per line
\usepackage{framed}
\definecolor{shadecolor}{RGB}{248,248,248}
\newenvironment{Shaded}{\begin{snugshade}}{\end{snugshade}}
\newcommand{\KeywordTok}[1]{\textcolor[rgb]{0.13,0.29,0.53}{\textbf{#1}}}
\newcommand{\DataTypeTok}[1]{\textcolor[rgb]{0.13,0.29,0.53}{#1}}
\newcommand{\DecValTok}[1]{\textcolor[rgb]{0.00,0.00,0.81}{#1}}
\newcommand{\BaseNTok}[1]{\textcolor[rgb]{0.00,0.00,0.81}{#1}}
\newcommand{\FloatTok}[1]{\textcolor[rgb]{0.00,0.00,0.81}{#1}}
\newcommand{\ConstantTok}[1]{\textcolor[rgb]{0.00,0.00,0.00}{#1}}
\newcommand{\CharTok}[1]{\textcolor[rgb]{0.31,0.60,0.02}{#1}}
\newcommand{\SpecialCharTok}[1]{\textcolor[rgb]{0.00,0.00,0.00}{#1}}
\newcommand{\StringTok}[1]{\textcolor[rgb]{0.31,0.60,0.02}{#1}}
\newcommand{\VerbatimStringTok}[1]{\textcolor[rgb]{0.31,0.60,0.02}{#1}}
\newcommand{\SpecialStringTok}[1]{\textcolor[rgb]{0.31,0.60,0.02}{#1}}
\newcommand{\ImportTok}[1]{#1}
\newcommand{\CommentTok}[1]{\textcolor[rgb]{0.56,0.35,0.01}{\textit{#1}}}
\newcommand{\DocumentationTok}[1]{\textcolor[rgb]{0.56,0.35,0.01}{\textbf{\textit{#1}}}}
\newcommand{\AnnotationTok}[1]{\textcolor[rgb]{0.56,0.35,0.01}{\textbf{\textit{#1}}}}
\newcommand{\CommentVarTok}[1]{\textcolor[rgb]{0.56,0.35,0.01}{\textbf{\textit{#1}}}}
\newcommand{\OtherTok}[1]{\textcolor[rgb]{0.56,0.35,0.01}{#1}}
\newcommand{\FunctionTok}[1]{\textcolor[rgb]{0.00,0.00,0.00}{#1}}
\newcommand{\VariableTok}[1]{\textcolor[rgb]{0.00,0.00,0.00}{#1}}
\newcommand{\ControlFlowTok}[1]{\textcolor[rgb]{0.13,0.29,0.53}{\textbf{#1}}}
\newcommand{\OperatorTok}[1]{\textcolor[rgb]{0.81,0.36,0.00}{\textbf{#1}}}
\newcommand{\BuiltInTok}[1]{#1}
\newcommand{\ExtensionTok}[1]{#1}
\newcommand{\PreprocessorTok}[1]{\textcolor[rgb]{0.56,0.35,0.01}{\textit{#1}}}
\newcommand{\AttributeTok}[1]{\textcolor[rgb]{0.77,0.63,0.00}{#1}}
\newcommand{\RegionMarkerTok}[1]{#1}
\newcommand{\InformationTok}[1]{\textcolor[rgb]{0.56,0.35,0.01}{\textbf{\textit{#1}}}}
\newcommand{\WarningTok}[1]{\textcolor[rgb]{0.56,0.35,0.01}{\textbf{\textit{#1}}}}
\newcommand{\AlertTok}[1]{\textcolor[rgb]{0.94,0.16,0.16}{#1}}
\newcommand{\ErrorTok}[1]{\textcolor[rgb]{0.64,0.00,0.00}{\textbf{#1}}}
\newcommand{\NormalTok}[1]{#1}
\usepackage{graphicx,grffile}
\makeatletter
\def\maxwidth{\ifdim\Gin@nat@width>\linewidth\linewidth\else\Gin@nat@width\fi}
\def\maxheight{\ifdim\Gin@nat@height>\textheight\textheight\else\Gin@nat@height\fi}
\makeatother
% Scale images if necessary, so that they will not overflow the page
% margins by default, and it is still possible to overwrite the defaults
% using explicit options in \includegraphics[width, height, ...]{}
\setkeys{Gin}{width=\maxwidth,height=\maxheight,keepaspectratio}
\IfFileExists{parskip.sty}{%
\usepackage{parskip}
}{% else
\setlength{\parindent}{0pt}
\setlength{\parskip}{6pt plus 2pt minus 1pt}
}
\setlength{\emergencystretch}{3em}  % prevent overfull lines
\providecommand{\tightlist}{%
  \setlength{\itemsep}{0pt}\setlength{\parskip}{0pt}}
\setcounter{secnumdepth}{0}
% Redefines (sub)paragraphs to behave more like sections
\ifx\paragraph\undefined\else
\let\oldparagraph\paragraph
\renewcommand{\paragraph}[1]{\oldparagraph{#1}\mbox{}}
\fi
\ifx\subparagraph\undefined\else
\let\oldsubparagraph\subparagraph
\renewcommand{\subparagraph}[1]{\oldsubparagraph{#1}\mbox{}}
\fi

%%% Use protect on footnotes to avoid problems with footnotes in titles
\let\rmarkdownfootnote\footnote%
\def\footnote{\protect\rmarkdownfootnote}

%%% Change title format to be more compact
\usepackage{titling}

% Create subtitle command for use in maketitle
\newcommand{\subtitle}[1]{
  \posttitle{
    \begin{center}\large#1\end{center}
    }
}

\setlength{\droptitle}{-2em}
  \title{Chapter 2 - The Causes and Quantification of Population Vulnerability}
  \pretitle{\vspace{\droptitle}\centering\huge}
  \posttitle{\par}
  \author{Mairin Deith}
  \preauthor{\centering\large\emph}
  \postauthor{\par}
  \date{}
  \predate{}\postdate{}


\begin{document}
\maketitle

=====

\section{Chapter 2 summary}\label{chapter-2-summary}

Chapter 2 of Morris and Doak's \emph{Quantitative Conservation Biology:
Theory and Practice of Population Vulnerability Analysis} discusses how
a population's vulnerability is influenced by factors like temporal
variability (e.g.~environmental stochasticity, demographic
stochasticity, and catastrophic/bonanza events). All of these factors
can negatively impact overall population growth by introducing
variability in the population's vital rates. As a result of variability,
populations can decrease in size even if the arithmetic mean of
population growth indicates positive growth (i.e. \(\lambda_A > 1\)).

The effects of temporal variability are magnified as we project further
forward in time.

\section{Box 2.1 - Simulating population trajectories under temporal
variability}\label{box-2.1---simulating-population-trajectories-under-temporal-variability}

This example model uses the basic equation of population growth,
\(N_{t+1} = \lambda_t N_t\) (Eq'n 2.1), but randomly draws the value of
\(\lambda_t\) from a set of observed values of \(\lambda\). While Morris
and Doak provide their code in MATLAB, I have translated it to a free
and open-source statistical language, R, below - there are also comments
to indicate which parts of the MATLAB code correspond to R commands. All
variable names are kept the same from the original MATLAB code, only the
functions and code structure have been modified.

\begin{Shaded}
\begin{Highlighting}[]
\NormalTok{### PROGRAM RandDraw.m, translated to RandDraw.R}
\NormalTok{### Performs multiple simulations of discrete exponential growth trajectores using a set of observed lambda values}

\NormalTok{### Load R libraries}
\CommentTok{# This code only uses three optional libraries, ggplot2 and the tidyverse packages tidyr and dplyr}
\CommentTok{# These packages are only used to make the plots produced by population simulations, but are not necessary for the actual simulations. }
\CommentTok{# If you'd rather not install and load these libraries, it won't affect the population simulation steps}
\KeywordTok{library}\NormalTok{(ggplot2)}
\KeywordTok{library}\NormalTok{(tidyverse)}
\end{Highlighting}
\end{Shaded}

\begin{verbatim}
## Warning: package 'tidyverse' was built under R version 3.4.4
\end{verbatim}

\begin{verbatim}
## -- Attaching packages ------------------------------------------------------ tidyverse 1.2.1 --
\end{verbatim}

\begin{verbatim}
## v tibble  1.4.2          v purrr   0.2.4     
## v tidyr   0.8.1          v dplyr   0.7.4.9002
## v readr   1.1.1          v stringr 1.2.0     
## v tibble  1.4.2          v forcats 0.3.0
\end{verbatim}

\begin{verbatim}
## Warning: package 'tidyr' was built under R version 3.4.4
\end{verbatim}

\begin{verbatim}
## Warning: package 'readr' was built under R version 3.4.4
\end{verbatim}

\begin{verbatim}
## Warning: package 'forcats' was built under R version 3.4.4
\end{verbatim}

\begin{verbatim}
## -- Conflicts --------------------------------------------------------- tidyverse_conflicts() --
## x dplyr::filter() masks stats::filter()
## x dplyr::lag()    masks stats::lag()
\end{verbatim}

\begin{Shaded}
\begin{Highlighting}[]
\NormalTok{### SIMULATION PARAMETERS}
\CommentTok{# The observed lambda values to sample from. Each has equal draw probability.}
\NormalTok{lams <-}\StringTok{ }\KeywordTok{c}\NormalTok{(}\FloatTok{1.00}\NormalTok{, }\FloatTok{1.98}\NormalTok{, }\FloatTok{1.02}\NormalTok{, }\FloatTok{0.92}\NormalTok{, }\FloatTok{0.53}\NormalTok{) }
\CommentTok{# Original population size at t=0}
\NormalTok{numzero <-}\StringTok{ }\DecValTok{29}
\CommentTok{# Number of years to simulate }
\NormalTok{tmax <-}\StringTok{ }\DecValTok{100}
\CommentTok{# How many replicates of the time series should be created}
\NormalTok{numreps <-}\StringTok{ }\DecValTok{100}
\CommentTok{# We ignore 'outname' in R because our results will exist in our R session}

\NormalTok{### MODEL}
\CommentTok{# This replaces the rand(...) MATLAB command, sets the seed for the random number generator}
\CommentTok{# SN: Unlike MATLAB, R does not require setting the seed for randomization, but setting it to 123 will ensure that your results match the ones I generated here}
\KeywordTok{set.seed}\NormalTok{(}\DataTypeTok{seed=}\DecValTok{123}\NormalTok{) }
\NormalTok{numlambs <-}\StringTok{ }\KeywordTok{length}\NormalTok{(lams) }\CommentTok{# How many lambda values are there?}

\CommentTok{# To generate a matrix of lambdas to use for each year/simulation:}
\CommentTok{#    a. Create a 100x100 matrix (one index for each value of 1:tmax and 1:numreps}
\CommentTok{#    b. For each value in the matrix, randomly sample a value of lambda (with replacement) from the lams list}
\CommentTok{# This combines the commands that create intvalues and lambdas in the original M+D MATLAB code}
\NormalTok{lambdas <-}\StringTok{ }\KeywordTok{matrix}\NormalTok{(}\DataTypeTok{nrow=}\NormalTok{tmax, }\DataTypeTok{ncol=}\NormalTok{numreps, }\DataTypeTok{data=}\KeywordTok{sample}\NormalTok{(lams, tmax}\OperatorTok{*}\NormalTok{numreps, }\DataTypeTok{replace=}\NormalTok{T))}
\KeywordTok{dim}\NormalTok{(lambdas) }\CommentTok{# Just to check if it matches our expectations of size and values}
\end{Highlighting}
\end{Shaded}

\begin{verbatim}
## [1] 100 100
\end{verbatim}

\begin{Shaded}
\begin{Highlighting}[]
\KeywordTok{unique}\NormalTok{(}\KeywordTok{as.numeric}\NormalTok{(lambdas))}
\end{Highlighting}
\end{Shaded}

\begin{verbatim}
## [1] 1.98 0.92 1.02 0.53 1.00
\end{verbatim}

\begin{Shaded}
\begin{Highlighting}[]
\CommentTok{# Now we need to initialize population size}
\NormalTok{initpopsize <-}\StringTok{ }\KeywordTok{rep}\NormalTok{(numzero, numreps) }\CommentTok{# initpopsize is a length-100 vector filled with 29, our starting population size}

\CommentTok{# For each year of the simulation, multiply the current population size by the lambdas*last year's population size}
\CommentTok{#   a) Initialize the popsizes matrix with NaN values - this will be populated later}
\CommentTok{#   b) For each year, calculate the new population size following deterministic growth with that year's lambda_t}
\CommentTok{# SN: There is likely a faster/more code-efficient way to do this; please raise an Issue on GitHub if you know a better solution}
\NormalTok{popsizes <-}\StringTok{ }\KeywordTok{matrix}\NormalTok{(}\DataTypeTok{nrow=}\NormalTok{tmax, }\DataTypeTok{ncol=}\NormalTok{numreps, }\DataTypeTok{data=}\OtherTok{NaN}\NormalTok{)}
\NormalTok{popsizes[}\DecValTok{1}\NormalTok{,] <-}\StringTok{ }\NormalTok{initpopsize }\CommentTok{# The first row of our simulation (i.e. our first year) is 29 for all replications}
\CommentTok{# Now iterate over years to calculate the new population size}
\CommentTok{# If the previous year's population is less than one, enforce a new population size of 0 }
\CommentTok{#    This ensures that we don't see populations with less than 1 individual (e.g. 0.5) rebuilding the population}
\ControlFlowTok{for}\NormalTok{(y }\ControlFlowTok{in} \DecValTok{2}\OperatorTok{:}\NormalTok{tmax)\{}
\NormalTok{  lambdas.year <-}\StringTok{ }\NormalTok{lambdas[y}\OperatorTok{-}\DecValTok{1}\NormalTok{,]}
  \CommentTok{# Temporary vector indicating whether the population is extinct in the past year or not}
\NormalTok{  notextinct.tmp <-}\StringTok{ }\NormalTok{(popsizes[y}\OperatorTok{-}\DecValTok{1}\NormalTok{,]}\OperatorTok{>=}\DecValTok{1}\NormalTok{)}\OperatorTok{*}\DecValTok{1}
\NormalTok{  popsizes[y,] <-}\StringTok{ }\NormalTok{popsizes[y}\OperatorTok{-}\DecValTok{1}\NormalTok{,]}\OperatorTok{*}\NormalTok{lambdas.year}\OperatorTok{*}\NormalTok{notextinct.tmp}
\NormalTok{\}}

\CommentTok{# Let's check on how the first few years of simulation look in the first 10 replications}
\CommentTok{# after giving the matrix some better column and row names}
\KeywordTok{colnames}\NormalTok{(popsizes) <-}\StringTok{ }\KeywordTok{paste0}\NormalTok{(}\StringTok{'rep'}\NormalTok{,}\DecValTok{1}\OperatorTok{:}\NormalTok{numreps)}
\CommentTok{# rownames(popsizes) <- paste0('y',1:tmax)}

\KeywordTok{head}\NormalTok{(popsizes[,}\DecValTok{1}\OperatorTok{:}\DecValTok{10}\NormalTok{])}
\end{Highlighting}
\end{Shaded}

\begin{verbatim}
##          rep1     rep2     rep3     rep4      rep5     rep6     rep7
## [1,] 29.00000 29.00000 29.00000 29.00000 29.000000  29.0000  29.0000
## [2,] 57.42000 29.58000 57.42000 26.68000 15.370000  57.4200  57.4200
## [3,] 52.82640 58.56840 30.43260 26.68000 15.370000 113.6916  52.8264
## [4,] 53.88293 59.73977 27.99799 24.54560  8.146100 225.1094 104.5963
## [5,] 28.55795 31.66208 28.55795 22.58195  8.309022 225.1094 207.1006
## [6,] 15.13571 32.29532 29.12911 20.77540 16.451864 445.7165 207.1006
##          rep8     rep9     rep10
## [1,] 29.00000  29.0000 29.000000
## [2,] 15.37000  29.5800 15.370000
## [3,] 30.43260  58.5684 15.677400
## [4,] 30.43260  58.5684  8.309022
## [5,] 16.12928  58.5684  8.475202
## [6,] 31.93597 115.9654  7.797186
\end{verbatim}

We can already see the impact that a variable \(\lambda_t\) can have on
population growth dynamics.

\begin{Shaded}
\begin{Highlighting}[]
\CommentTok{# Now identify situations in which the population went extinct}
\NormalTok{notextinct <-}\StringTok{ }\NormalTok{(popsizes}\OperatorTok{>=}\DecValTok{1}\NormalTok{)}\OperatorTok{*}\DecValTok{1}
\KeywordTok{table}\NormalTok{(notextinct)}
\end{Highlighting}
\end{Shaded}

\begin{verbatim}
## notextinct
##    0    1 
## 1957 8043
\end{verbatim}

In 1957 instances, we saw extinction.

\begin{Shaded}
\begin{Highlighting}[]
\CommentTok{# Find the biggest population size}
\NormalTok{maxN <-}\StringTok{ }\FloatTok{0.5}\OperatorTok{*}\KeywordTok{max}\NormalTok{(popsizes)}
\NormalTok{maxN}
\end{Highlighting}
\end{Shaded}

\begin{verbatim}
## [1] 24916.48
\end{verbatim}

Oh boy\ldots{}that's a really big number. Let's set a cutoff of 200,
like in the textbook.

\begin{Shaded}
\begin{Highlighting}[]
\CommentTok{# Calculate the geometric mean for lambda}
\NormalTok{stochL <-}\StringTok{ }\KeywordTok{prod}\NormalTok{(lams)}\OperatorTok{^}\NormalTok{(}\DecValTok{1}\OperatorTok{/}\NormalTok{numlambs)}
\CommentTok{# Calculate the expected population at tmax}
\NormalTok{expectedNt <-}\StringTok{ }\NormalTok{numzero}\OperatorTok{*}\NormalTok{(stochL}\OperatorTok{^}\NormalTok{tmax)}

\KeywordTok{cat}\NormalTok{(}\StringTok{'Stochastic lambda:'}\NormalTok{, stochL); }\KeywordTok{cat}\NormalTok{(}\StringTok{'}\CharTok{\textbackslash{}n}\StringTok{Expected N at t=tmax:'}\NormalTok{, expectedNt) }
\end{Highlighting}
\end{Shaded}

\begin{verbatim}
## Stochastic lambda: 0.9969326
\end{verbatim}

\begin{verbatim}
## 
## Expected N at t=tmax: 21.32943
\end{verbatim}

Overall, we should expect the population to decrease over time (compared
to the arithmetic mean, 1.09)

\begin{Shaded}
\begin{Highlighting}[]
\CommentTok{# We're going to skip writing to the worksheet and plot the population sizes using ggplot}
\CommentTok{# Find the average population size for each year, averaging across replications}
\NormalTok{averageN <-}\StringTok{ }\KeywordTok{data.frame}\NormalTok{(}\DataTypeTok{averagePop=}\KeywordTok{rowMeans}\NormalTok{(popsizes[,}\DecValTok{1}\OperatorTok{:}\DecValTok{100}\NormalTok{]),}
                       \DataTypeTok{year=}\DecValTok{1}\OperatorTok{:}\NormalTok{tmax)}

\CommentTok{# Let's get to the histogram in a moment, first let's see a time series}
\CommentTok{#    To do this, I'm going to first rearrange the dataframe using tidyr}
\NormalTok{popsizes.mod <-}\StringTok{ }\KeywordTok{as.data.frame}\NormalTok{(popsizes) }\OperatorTok\StringTok{ }
\StringTok{  }\KeywordTok{gather}\NormalTok{(}\DataTypeTok{key =}\NormalTok{ rep, }\DataTypeTok{value =}\NormalTok{ populationSize) }\OperatorTok\StringTok{ }
\StringTok{  }\KeywordTok{mutate}\NormalTok{(}\DataTypeTok{year =} \KeywordTok{rep}\NormalTok{(}\DecValTok{1}\OperatorTok{:}\NormalTok{tmax,numreps))}
  \CommentTok{# I need to add the +1 here to account for the annual mean values}

\KeywordTok{ggplot}\NormalTok{(}\DataTypeTok{data=}\KeywordTok{as.data.frame}\NormalTok{(popsizes.mod), }\KeywordTok{aes}\NormalTok{(}\DataTypeTok{x=}\NormalTok{year, }\DataTypeTok{y=}\NormalTok{populationSize, }\DataTypeTok{color=}\NormalTok{rep)) }\OperatorTok{+}\StringTok{ }
\StringTok{  }\KeywordTok{geom_line}\NormalTok{(}\DataTypeTok{alpha=}\FloatTok{0.7}\NormalTok{) }\OperatorTok{+}\StringTok{ }\CommentTok{# plot out our replicated series}
\StringTok{  }\KeywordTok{geom_line}\NormalTok{(}\DataTypeTok{data=}\NormalTok{averageN, }\KeywordTok{aes}\NormalTok{(}\DataTypeTok{y=}\NormalTok{averagePop, }\DataTypeTok{x=}\NormalTok{year), }\DataTypeTok{color=}\StringTok{'black'}\NormalTok{, }\DataTypeTok{size=}\FloatTok{1.2}\NormalTok{) }\OperatorTok{+}\StringTok{ }\CommentTok{# Plot the average size overtop}
\StringTok{  }\KeywordTok{ylim}\NormalTok{(}\DecValTok{0}\NormalTok{,}\DecValTok{2000}\NormalTok{) }\OperatorTok{+}\StringTok{ }\CommentTok{# limit our y axis from 0 to 2000}
\StringTok{  }\KeywordTok{theme}\NormalTok{(}\DataTypeTok{legend.position =} \StringTok{'none'}\NormalTok{) }\CommentTok{# get rid of the legend}
\end{Highlighting}
\end{Shaded}

\begin{verbatim}
## Warning: Removed 183 rows containing missing values (geom_path).
\end{verbatim}

\begin{verbatim}
## Warning: Removed 1 rows containing missing values (geom_path).
\end{verbatim}

\includegraphics{Chap2_CausesConsequencesPopVulnerability_files/figure-latex/unnamed-chunk-4-1.pdf}
Now let's make those histograms after bounding our data to be below 2000
individuals (like the y-axis above).

\begin{Shaded}
\begin{Highlighting}[]
\NormalTok{bounded.popsizes <-}\StringTok{ }\NormalTok{popsizes.mod}
\NormalTok{bounded.popsizes}\OperatorTok{$}\NormalTok{populationSize[}\KeywordTok{which}\NormalTok{(bounded.popsizes}\OperatorTok{$}\NormalTok{populationSize}\OperatorTok{>}\DecValTok{2000}\NormalTok{)] <-}\StringTok{ }\DecValTok{2000}

\CommentTok{# Let's check that worked:}
\KeywordTok{summary}\NormalTok{(bounded.popsizes)}
\end{Highlighting}
\end{Shaded}

\begin{verbatim}
##      rep            populationSize          year       
##  Length:10000       Min.   :   0.000   Min.   :  1.00  
##  Class :character   1st Qu.:   3.346   1st Qu.: 25.75  
##  Mode  :character   Median :  23.334   Median : 50.50  
##                     Mean   : 209.800   Mean   : 50.50  
##                     3rd Qu.: 102.611   3rd Qu.: 75.25  
##                     Max.   :2000.000   Max.   :100.00
\end{verbatim}

\begin{Shaded}
\begin{Highlighting}[]
\CommentTok{# Good, the max value is 2000.}
\end{Highlighting}
\end{Shaded}

\begin{Shaded}
\begin{Highlighting}[]
\CommentTok{# Histogram for the first 5 years}
\KeywordTok{ggplot}\NormalTok{(}\DataTypeTok{data=}\NormalTok{bounded.popsizes[}\KeywordTok{which}\NormalTok{(bounded.popsizes}\OperatorTok{$}\NormalTok{year }\OperatorTok\StringTok{ }\KeywordTok{c}\NormalTok{(}\DecValTok{1}\OperatorTok{:}\DecValTok{5}\NormalTok{)),], }\KeywordTok{aes}\NormalTok{(}\DataTypeTok{x=}\NormalTok{populationSize)) }\OperatorTok{+}
\StringTok{  }\KeywordTok{geom_histogram}\NormalTok{()}
\end{Highlighting}
\end{Shaded}

\begin{verbatim}
## `stat_bin()` using `bins = 30`. Pick better value with `binwidth`.
\end{verbatim}

\includegraphics{Chap2_CausesConsequencesPopVulnerability_files/figure-latex/unnamed-chunk-6-1.pdf}

\begin{Shaded}
\begin{Highlighting}[]
\CommentTok{# For the first 20 years}
\KeywordTok{ggplot}\NormalTok{(}\DataTypeTok{data=}\NormalTok{bounded.popsizes[}\KeywordTok{which}\NormalTok{(bounded.popsizes}\OperatorTok{$}\NormalTok{year }\OperatorTok\StringTok{ }\KeywordTok{c}\NormalTok{(}\DecValTok{1}\OperatorTok{:}\DecValTok{20}\NormalTok{)),], }\KeywordTok{aes}\NormalTok{(}\DataTypeTok{x=}\NormalTok{populationSize)) }\OperatorTok{+}
\StringTok{  }\KeywordTok{geom_histogram}\NormalTok{()}
\end{Highlighting}
\end{Shaded}

\begin{verbatim}
## `stat_bin()` using `bins = 30`. Pick better value with `binwidth`.
\end{verbatim}

\includegraphics{Chap2_CausesConsequencesPopVulnerability_files/figure-latex/unnamed-chunk-7-1.pdf}

\begin{Shaded}
\begin{Highlighting}[]
\CommentTok{# For the full 100 years}
\KeywordTok{ggplot}\NormalTok{(}\DataTypeTok{data=}\NormalTok{bounded.popsizes, }\KeywordTok{aes}\NormalTok{(}\DataTypeTok{x=}\NormalTok{populationSize)) }\OperatorTok{+}
\StringTok{  }\KeywordTok{geom_histogram}\NormalTok{()}
\end{Highlighting}
\end{Shaded}

\begin{verbatim}
## `stat_bin()` using `bins = 30`. Pick better value with `binwidth`.
\end{verbatim}

\includegraphics{Chap2_CausesConsequencesPopVulnerability_files/figure-latex/unnamed-chunk-8-1.pdf}

That's the end of the practical example in M+D's PVA textbook. The
remainder of the chapter discusses the influence of negative density
dependence, positive density dependence (i.e.~Allee effects), and
genetic factors on population growth and viability.

\section{Notes for the remainder of the
chapter}\label{notes-for-the-remainder-of-the-chapter}

\subsection{Density dependence}\label{density-dependence}

\subsubsection{Negative density
dependence}\label{negative-density-dependence}

\begin{itemize}
\item
  \textbf{Density dependence} = a change in individual performance as
  the size or density of the population changes; embodied by the
  logistic growth equation \(N_{t+1} = N_t + rN_t(1-\frac{N_t}{K})\) \#
  CHECK THE ABOVE EQUATION
\item
  can cause PVA to over or underestimate the true viability of a
  population depending on how growth rates respond to density
\item
  relationship is typically nonlinear - effects may not manifest until
  high population densities
\item
  if we assume linear negative density dependence - greater probability
  of species extinction; this is because \textbf{negative density
  dependence maintains the population at a low level} because higher
  densities = lower fecundity
\item
  at the same time, because fecundity is higher at low densitites, the
  risk of extinction \textbf{can also decrease the face of negative
  density dependence} as a result of reproductive compensation
\end{itemize}

In contrast, if density dependence is introduced only at a high
densities, there will be \textbf{no added protection of density
dependence as fecundity is not increased at low densities}. If PVA
assumed perfect density dependence, this would artificially increase our
confidence that the population will persist.

As a result, many PVAs ignore density dependence although many methods
exist to estimate and include negative density dependence in PVA (or use
a simple population ceiling). It is usually best to include no density
dependent effects if careful testing of population data indicate no
density dependence \emph{or} create a range of models with different
strengths/forms of density dependence.

\subsubsection{Positive density effects/Allee
effects}\label{positive-density-effectsallee-effects}

\begin{itemize}
\tightlist
\item
  \textbf{increase in population growth rate as size increases} - arise
  from improved mating success, group defense*, or group foraging* as
  density increases (majority of evidence from sessile plants)
\end{itemize}

* - only important for some species, while all affected by mate success

\begin{itemize}
\tightlist
\item
  \textbf{+ DD only rarely detected with census data} - e.g.~using
  quadratic regression between per-capita population growth rate and N
  in birds/mammals with census data @ low sizes - expected to see a
  hump-shaped curve
\item
  neither analysis found any evidence for + DD
\item
  Myers et al. looked for + DD in fish - fit BH models with and without
  allee effects - in only 3/128 stocks did allee effects fit the
  recruitment data better than BH without
\item
  logistic regression of butterflies showed proportion of small
  populations that grew or remained constant from \emph{t} to \emph{t+1}
  was positively related to population size at \emph{t} - looked at
  replication of multiple populations in space rather than a single
  population over time
\item
  \textbf{thus we do not have a good idea of how Allee effects operate
  or how strong they are, even though they are almost certainly
  operating}
\item
  we can incorporate these into PVas, but the limits of our data can
  thwart inclusion of positive density effects; best solution often to
  set a quasi-extinction threshold high enough to avoid Allee effects
\item
  But what this level is can be confusing (e.g.~passenger pigeons may
  have been 10,000; but for some plants it is more like 20) - again,
  best to run a set of PVA models and compare their outcomes
\end{itemize}

\subsection{Genetic factors}\label{genetic-factors}

At low population sizes, genetic factors are especially important.
Inclusion in PVA requires \textbf{estimates of 2 sets of processes}:

\begin{enumerate}
\def\labelenumi{\arabic{enumi}.}
\tightlist
\item
  Rate at which heterozygosity (genetic diversity, \emph{probability
  that for the average locus in the average individual in the
  population, there will be different alleles}) will be lost by a pop'n
  of certain size
\end{enumerate}

\begin{itemize}
\tightlist
\item
  This is often estimated as the change in \textbf{inbreeding level per
  generation} (\emph{inbreeding = average probability that an
  individuals two copies of a gene are `identical by descent'})
\end{itemize}

\begin{enumerate}
\def\labelenumi{\arabic{enumi}.}
\setcounter{enumi}{1}
\tightlist
\item
  What are the consequences of how fast the loss of heterozygosity will
  be?
\end{enumerate}

\begin{itemize}
\tightlist
\item
  decrease in individual performance as a result of inbreeding =
  \textbf{inbreeding depression}, commonly measured as a percent change
  in some vital rate (e.g.~survival, reproduction) with a loss in
  heterozygosity/increase in inbreeding level
\end{itemize}

Because it can be very difficult to estimate the influence/extent of
breeding depressions on wild species (domesticated easier), some argue
that if pop sizes are low enough to suffer from inbreeding depressions,
they are susceptible to demographic stochasticity anyway. \emph{But some
studies show synergistic effects with demography to impact population
health}.

This feeds the \textbf{extinction vortex}:

\begin{enumerate}
\def\labelenumi{\arabic{enumi}.}
\tightlist
\item
  e.g.~Environmental variation = low levels for one generation
\item
  = small increase in inbreeding
\item
  = slightly decreased vital rates compared to what it was before
\item
  If the population drops again, it will stay there longer due to poorer
  per-capita performance
\item
  = increased susceptability to env and dem stochasticity
\end{enumerate}

While \textbf{purging} can eliminate deleterious alleles through natural
selection and reduce the severity of subsequent bottlenecks. \emph{But
the rate of purging of deleterious alleles depends on the rate of
inbreeding and the genetic mechanism that creates the inbreeding
depression}.

Generally, do not include genetic effects in PVA - again, use a
quasi-extinction threshold (e.g. \(N_e = 50\), effective population size
\textgreater{} 50 individuals tends to result in only 1\% loss of
genetic diversity per generation).
\(Ne \approx \frac{N}{2} or \frac{2N}{3}\) where \(N\) is the number of
\textbf{reproductive adults}.

Still basically ignores inbreeding problems, but it's better to not make
wild guesses.

\section{Quantifying population
viability}\label{quantifying-population-viability}

\subsection{Viability metrics related to
Pr(quasi-extinction)}\label{viability-metrics-related-to-prquasi-extinction}

\textbf{No good PVA should evaluate the risk of total extinction} - the
process is complicated, and we shouldn't assume to know the true
population dynamics that might influence ultimate extinction. Therefore
it's more appropriate to measure \textbf{quasi-extinction thresholds} -
the number of females that should be considered a bare minimum - below
this level, the population is likely to be critically and immediately
imperiled.

Based on dem. stoch, this could be \textasciitilde{}20 reproductive
individuals; genetic arguments favour \textgreater{}100 breeders
minimum. \emph{Lower values may be required if the population is already
below some preferable extinction threshold - e.g.~African elephant's
QE-threshold is 1}.

\begin{Shaded}
\begin{Highlighting}[]
\KeywordTok{plot}\NormalTok{(}\KeywordTok{dgamma}\NormalTok{(}\DataTypeTok{x=}\KeywordTok{seq}\NormalTok{(}\DecValTok{0}\NormalTok{,}\DecValTok{100}\NormalTok{),}\DataTypeTok{shape=}\DecValTok{2}\NormalTok{, }\DataTypeTok{scale=}\DecValTok{8}\NormalTok{),}\DataTypeTok{type =} \StringTok{'l'}\NormalTok{, }\DataTypeTok{ylab=}\StringTok{'Probability density function'}\NormalTok{,}\DataTypeTok{xlab=}\StringTok{'Years into the future'}\NormalTok{, }\DataTypeTok{main=}\StringTok{'Example extinction risk'}\NormalTok{)}
\KeywordTok{abline}\NormalTok{(}\DataTypeTok{v =} \KeywordTok{which}\NormalTok{(}\KeywordTok{dgamma}\NormalTok{(}\DataTypeTok{x=}\KeywordTok{seq}\NormalTok{(}\DecValTok{0}\NormalTok{,}\DecValTok{100}\NormalTok{),}\DataTypeTok{shape=}\DecValTok{2}\NormalTok{, }\DataTypeTok{scale=}\DecValTok{8}\NormalTok{) }\OperatorTok{==}\StringTok{ }\KeywordTok{max}\NormalTok{(}\KeywordTok{dgamma}\NormalTok{(}\DataTypeTok{x=}\KeywordTok{seq}\NormalTok{(}\DecValTok{0}\NormalTok{,}\DecValTok{100}\NormalTok{),}\DataTypeTok{shape=}\DecValTok{2}\NormalTok{, }\DataTypeTok{scale=}\DecValTok{8}\NormalTok{))), }\DataTypeTok{lty=}\DecValTok{2}\NormalTok{)}
\end{Highlighting}
\end{Shaded}

\includegraphics{Chap2_CausesConsequencesPopVulnerability_files/figure-latex/unnamed-chunk-9-1.pdf}
Many PVAs show this relationship; extinction is most likely at the
beginning - if it hasn't gone extinct in the first 10 years, there is
less likelihood that it will eventually go extinct.

\begin{Shaded}
\begin{Highlighting}[]
\KeywordTok{plot}\NormalTok{(}\KeywordTok{pgamma}\NormalTok{(}\DataTypeTok{q=}\KeywordTok{seq}\NormalTok{(}\DecValTok{0}\NormalTok{,}\DecValTok{100}\NormalTok{),}\DataTypeTok{shape=}\DecValTok{2}\NormalTok{, }\DataTypeTok{scale=}\DecValTok{20}\NormalTok{),}\DataTypeTok{type =} \StringTok{'l'}\NormalTok{, }\DataTypeTok{ylab=}\StringTok{'Cumulative distribution function'}\NormalTok{,}\DataTypeTok{xlab=}\StringTok{'Years into the future'}\NormalTok{, }\DataTypeTok{main=}\StringTok{'Example extinction risk'}\NormalTok{)}
\KeywordTok{abline}\NormalTok{(}\DataTypeTok{h=}\FloatTok{0.5}\NormalTok{, }\DataTypeTok{lty=}\DecValTok{2}\NormalTok{)}
\end{Highlighting}
\end{Shaded}

\includegraphics{Chap2_CausesConsequencesPopVulnerability_files/figure-latex/unnamed-chunk-10-1.pdf}

The cumulative distribution function is much more applicable for PVA -
where we are interested what the probability of QE is \emph{at or before
some time}, not \emph{at some time}. It asymptotes to 1 because at some
point, the species will almost certainly go extinct.

\subsubsection{Valuable data from the probability distributions of
extinction
risk}\label{valuable-data-from-the-probability-distributions-of-extinction-risk}

\begin{enumerate}
\def\labelenumi{\arabic{enumi}.}
\tightlist
\item
  Probability of extinction \emph{by some given point}
\item
  Ultimate probability of extinction = probability that the QE threshold
  will be reached \emph{at any time}
\item
  Median time to extinction; \textbf{median} = point where there is a
  50\% cumulative probability (dashed line)
\item
  Mean time to extinction (not obvious from probability plots, but can
  be calculated from the data)
\item
  Mode of extinction time = peak of the probability distribution (dashed
  line)
\end{enumerate}

\textbf{1-3 are the most useful, 4-5 are the most commonly used but
least useful}. It is important to show probability of extinction within
some management timescale - not just a biological decision, but funding
deadlines, future threats, political changes, and other real-world
constraints. But because the choice can be somewhat subjective, it is
rarely used. Sad.

Together, the median time to extinction and the ultimate risk of
extinction are meaningful. Median = good description of extinction time
(half of the possible paths lead to extinction by this time), ultimate
is needed to show just how likely extinction at any time is. If the
probability of extinction ever happening is low, so is the median (this
can happen if a population is quite small and expected to grow, and if
extinctions occur they will occur early).

Mean time is almost always an overestimate, and the mode doesn't look at
cumulative distributions - considers only the instant at which
extinction is most likely to occur instantaneously, not cumulatively.
Mode is always shorter than median.

Overall estimates of extinction risk aren't particularly useful - all
species might eventually go extinct, but their extinction is over such a
long timescale that it isn't practical.

PVAs should \textbf{show the entire cumulative distribution of
extinction times, discuss the risk of extinction by certain time
horizons with biological/practical considerations}. The distribution is
far more useful than any summary statistics (\emph{accidental
Bayesian}).

\subsection{Viability metrics related to population growth
rate}\label{viability-metrics-related-to-population-growth-rate}

\textbf{Always use an estimate of stochastic growth!}

Growth rate may be a more important indicator of possible future
problems - e.g.~there are \textasciitilde{}150 female right whales, and
the population is declining as of 1999 (\(\lambda = 0.976\)). It is
likely to go extinct in 300 years, with near-zero risk over the next
100.

Another benefit - easier to estimate reliably with spotty/short-term
data.

Ultimately, there is a massive argument about whether small populations
are doomed as a result of environmental stochasticity (\emph{small
population paradigm}), or deterministic factors (\emph{declining
population paradigm}, e.g.~Caughley 1994). Obviously, stochasticity does
matter \textbf{a lot}, so let's not ignore it.

\subsection{Viability metrics related to population size and
number}\label{viability-metrics-related-to-population-size-and-number}

Typically, some target population size/level is known to be `safe'. This
automatically includes some assumptions about growth (\textgreater{}1),
probability of extinction is low once that size is achieved.

However, the target sizes are typically decided somewhat ad-hoc by
committees without data/sound science. e.g.~ESA - 28\% of recovery plans
set the population size for recovery \textbf{at or below the number of
individuals in existance at the time of writing, 37\% set the target at
or below the number at time of listing}. For plants, the goals
correlated very strongly with those in existence at the time of writing.

e.g.~US Marine Mammal Protection Act (MMPA) - on the assumption that a
carrying capacity exists for each population, protection is necessary at
60\% of capacity when it is classified as \textbf{depleted} (typically
more conservative than ESA listings). However, this is highly inflexible
- if 2 species exist at the same numbres, but one was once much more
common, they are treated as having different risk (which may not be the
case).

e.g.~IUCN - adopted criteria for endangerment that rely on estimates of
extinction risk, but these have been roughly translated into population
numbers, sizes, and trends so they can be applied without exhaustive
information.

But the criteria are complex - developed to be used with a variety of
kinds/qualities of data. Critically endangered species = those which
show either (a) Reductions of at least 80\% in the last 10 years/3
generations (whichever is longer), (b) an extent of occurence of less
than 100km2 or area of occupancy less than 10km2, (c) a population of
less than 250 individuals with evidence of decline, (d) popopulation of
less than 50 mature individuals, or (e) 50\% probability of extinction
in the wild in 10 years/3 generations.

If there is no possibility for doing a PVA, the simple numerical
criteria are useful. But larger populations can also be deeply imperiled
- current population size is a `second-best' option compared to PVA.
They are useful for triage, but must be done with the knowledge that
small stable populations may be at lower risk than larger populations
subjected to strong variability.

A better option = \textbf{perform an analysis of population dynamics} to
assess future stochastic population growth rate and risk of extinction.


\end{document}
